%%%%%%%%%%%%%%%%%%%%%%%%%%%%%%%%%%%%%%%%%
% Structured General Purpose Assignment
% LaTeX Template
%
% This template has been downloaded from:
% http://www.latextemplates.com
%
% Original author:
% Ted Pavlic (http://www.tedpavlic.com)
%
% Note:
% The \lipsum[#] commands throughout this template generate dummy text
% to fill the template out. These commands should all be removed when
% writing assignment content.
%
%%%%%%%%%%%%%%%%%%%%%%%%%%%%%%%%%%%%%%%%%

%----------------------------------------------------------------------------------
%   PACKAGES AND OTHER DOCUMENT CONFIGURATIONS
%----------------------------------------------------------------------------------

\documentclass{article}

\usepackage[utf8]{inputenc}
\usepackage[spanish]{babel}
\usepackage{fancyhdr} % Required for custom headers
\usepackage{lastpage} % Required to determine the last page for the footer
\usepackage{graphicx} % Required to insert images
\usepackage{tikz}
\usepackage[export]{adjustbox}
\usepackage{enumitem}
\usepackage{environ}
\usepackage{multicol}
\usepackage{hyperref}
\usepackage[font=small]{caption}
\usepackage{pifont}
\selectlanguage{spanish}
\addto\extrasspanish{%
    \def\figureautorefname{Figura}%
}
\newcommand{\myarrow}{\ding{223}}
\newcommand{\cmark}{\ding{51}}
\newcommand{\xmark}{\ding{55}}

% Margins
\topmargin=-0.45in
\evensidemargin=0in
\oddsidemargin=0in
\textwidth=6.5in
\textheight=9.0in
\headsep=0.25in

\linespread{1.1} % Line spacing

% Set up the header and footer
\pagestyle{fancy}
\lhead{\small \hmwkClass: \hmwkTitle} % Top left header
\chead{} % Top center header
\rhead{\small \hmwkAuthorName} % Top right header
\lfoot{} % Bottom left footer
\cfoot{} % Bottom center footer
\rfoot{Página\ \thepage\ de\ \pageref{LastPage}} % Bottom right footer
\renewcommand\headrulewidth{0.4pt} % Size of the header rule
\renewcommand\footrulewidth{0.4pt} % Size of the footer rule

\setlength\parindent{0pt} % Removes all indentation from paragraphs
\setlength{\multicolsep}{6.0pt plus 2.0pt minus 1.5pt} % 50% of original values

\newlength\widest
\makeatletter
\NewEnviron{ldescription}{%
    \vbox{%
        \global\setlength\widest{0pt}%
        \def\item[##1]{%
            \settowidth\@tempdima{\textbf{##1}}%
            \ifdim \@tempdima>\widest \global\setlength\widest{\@tempdima} \fi%
        }%
        \setbox0=\hbox{\BODY}%
    }
    \begin{description}[leftmargin=\dimexpr\widest+0.5em\relax,labelindent=0pt, labelwidth=\widest]
        \BODY
\end{description}%
}
\makeatother

%----------------------------------------------------------------------------------
%   NAME AND CLASS SECTION
%----------------------------------------------------------------------------------

\newcommand{\hmwkTitle}{Práctica\ 3} % Assignment title
\newcommand{\hmwkClass}{Sistemas Informáticos I} % Course/class
\newcommand{\hmwkClassTime}{10:30am} % Class/lecture time
\newcommand{\hmwkAuthorName}{\small Sergio Fuentes de Uña | Daniel Perdices Burrero} % Your name

%----------------------------------------------------------------------------------
%   TITLE PAGE
%----------------------------------------------------------------------------------

\title{
    \vspace{2in}
    \textmd{\textbf{\hmwkClass:\ \hmwkTitle}}\\
    \vspace{0.1in}
    \begin{center}
        \begin{tikzpicture}
            \begin{scope}
                \clip [rounded corners=12pt] (0,0) rectangle coordinate (centerpoint) (160pt,40pt);
                \node [inner sep=0pt] at (centerpoint) {\includegraphics[width=160pt]{betabet}};
            \end{scope}
        \end{tikzpicture}
    \end{center}
    \vspace{3in}
}

\author{\textbf{\hmwkAuthorName}}

%----------------------------------------------------------------------------------

\begin{document}

\maketitle

%----------------------------------------------------------------------------------
%   TABLE OF CONTENTS
%----------------------------------------------------------------------------------

%\setcounter{tocdepth}{1} % Uncomment this line if you don't want subsections listed in the ToC

\newpage
\tableofcontents
\newpage
\section{Análisis de la base de datos proporcionada}

\subsection{Esquema de las tablas}
Leyenda: \texttt{*}:\textit{primary key}, \texttt{+}:\textit{foreign key}

\begin{multicols}{2}
\begin{itemize}
    \item\texttt{customers}
        \begin{itemize}
            \item\texttt{customerid*} 
            \item\texttt{firstname}
            \item\texttt{lastname}
            \item\texttt{address1}
            \item\texttt{address2}
            \item\texttt{city}
            \item\texttt{state}
            \item\texttt{zip}
            \item\texttt{country}
            \item\texttt{region}
            \item\texttt{email}
            \item\texttt{phone}
            \item\texttt{creditcardtype}
            \item\texttt{creditcard}
            \item\texttt{creditcardexpiration}
            \item\texttt{username}
            \item\texttt{password}
            \item\texttt{age}
            \item\texttt{credit}
            \item\texttt{gender}
        \end{itemize}
    \item\texttt{clientorders}
        \begin{itemize}
            \item\texttt{customerid}
            \item\texttt{date}
            \item\texttt{orderid}
        \end{itemize}
    \columnbreak
    \item\texttt{clientbets}
        \begin{itemize}
            \item\texttt{customerid+}
            \item\texttt{optionid+}
            \item\texttt{betid+}
            \item\texttt{orderid}
            \item\texttt{bet}
            \item\texttt{ratio}
            \item\texttt{outcome}
        \end{itemize}
    \item\texttt{bets}
        \begin{itemize}
            \item\texttt{betid*}
            \item\texttt{betcloses}
            \item\texttt{category}
            \item\texttt{betdesc}
            \item\texttt{winneropt}
        \end{itemize}
    \item\texttt{options}
        \begin{itemize}
            \item\texttt{optionid*}
            \item\texttt{optiondesc}
            \item\texttt{categoria}
            \item\texttt{cuota}
        \end{itemize}
    \item\texttt{optionbet}
        \begin{itemize}
            \item\texttt{optionid+}
            \item\texttt{betid+}
            \item\texttt{ratio}
            \item\texttt{optiondesc}
        \end{itemize}
\end{itemize}
\end{multicols}
\newpage
\subsection{Organización de las tablas}
La base de datos consta de las siguientes tablas:
\begin{ldescription}
    \item[$\bullet$ \texttt{customers}]
        Clientes de la página de apuestas.
    \item[$\bullet$ \texttt{clientorders}]
        Carritos de los clientes.
    \item[$\bullet$ \texttt{clientbets}]
        Apuestas realizadas por los clientes.
    \item[$\bullet$ \texttt{bets}]
        Apuestas disponibles.
    \item[$\bullet$ \texttt{options}]
        Entidades deportivas.
    \item[$\bullet$ \texttt{optionbet}]
        Entidades concretas que se involucran en cada apuesta
\end{ldescription}
Estas tablas se encuentran relacionadas a traves del siguiente diagrama Entidad-Relación
\smallbreak
\begin{minipage}{\linewidth}
    \centering
    \captionsetup{type=figure}
    \includegraphics[width=\linewidth]{img/db_old.png}
    \caption{Diagrama E-R de la base de datos proporcionada}
    \label{fig:fig1}
\end{minipage}
\subsection{Diseño inicial de las tablas}
Con lo proporcionado, se pueden ver varias claras decisiones de diseño tomadas a la hora de elaborar la base de datos que se ha proporcionado:
\begin{enumerate}
    \item El atributo \texttt{customerid} de la tabla \texttt{clientbets} es redundante con el atributo de igual nombre de la tabla \texttt{clientorders}.
    \item El atributo \texttt{category} de la tabla \texttt{bets} y el atributo \texttt{categoria} de la tabla \texttt{options} son redundantes de igual manera, además, en la tabla las categorías se suelen repetir, por tanto se gasta mucho espacio de almacenamiento en guardar la misma cadena varias ocasiones.
    \item El atributo \texttt{optiondesc} del mismo modo aparece en dos tablas, \texttt{options} y \texttt{optionbet}, aunque se podría ver como el mismo.
\end{enumerate}
Muchas de estas repeticiones, especialmente la del \texttt{customerid} son bastante discutibles pues la repetición del dato mejora el rendimiento de la base de datos.
\subsection{Propuesta de diseño de las tablas}
Se han realizado los siguientes cambios en las tablas para normalizar la base de datos evitando futuras incosistencias. Se han añadido restricciones para evitar registros erroneos.
\begin{itemize}
    \item\texttt{customers:}
        \begin{itemize}
            \item Se agrega la restricción de email válido.
            \item Se agrega la restricción de edad válida.
            \item Se agrega la restricción de código postal válida.
            \item Se agrega la restricción de tarjeta de crédito válida.
            \item Se agrega la restricción de teléfono válido.
        \end{itemize}
    \item\texttt{clientorders:}
        \begin{itemize}
            \item Se agrega la restricción de orderid como primary key.
            \item Se agrega la restricción de customerid como foreign key.
            \item Se activa el borrado en cascada si se elimina el cliente (customerid).
        \end{itemize}
    \item\texttt{clientbets:}
        \begin{itemize}
            \item Se agrega la restricción de orderid como foreign key.
            \item Se elimina la columna customerid.
            \item Se activa el borrado en cascada si se elimina el carrito (orderid).
        \end{itemize}
    \item\texttt{bets}:
        \begin{itemize}
            \item Se elimina el campo category.
            \item Se agrega el campo categoryid como foreign key.
            \item Se agrega la restricción de winneropt como foreign key.
        \end{itemize}
    \item\texttt{options}
        \begin{itemize}
            \item Se elimina el campo categoria.
            \item Se agrega el campo categoryid como foreign key.
        \end{itemize}
    \item\texttt{optionbet}
        \begin{itemize}
            \item Se elimina la columna optiondesc.
        \end{itemize}
    \item\texttt{categories}
        \begin{itemize}
            \item Se agrega la tabla.
            \item Se agrega el campo categoryid como primary key.
            \item Se agrega el campo categorystring.
        \end{itemize}
\end{itemize}
\newpage
A continuación se muestra el diagrama E-R (sin atributos por legibilidad) del estado final.
\smallbreak
\begin{minipage}{\linewidth}
    \centering
    \captionsetup{type=figure}
    \includegraphics[width=\linewidth]{img/db_new.png}
    \caption{Diagrama E-R de la base de datos final}
    \label{fig:fig2}
\end{minipage}

Todos estos cambios los realiza el fichero \texttt{actualiza.sql}
\end{document}
