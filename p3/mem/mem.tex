%%%%%%%%%%%%%%%%%%%%%%%%%%%%%%%%%%%%%%%%%
% Structured General Purpose Assignment
% LaTeX Template
%
% This template has been downloaded from:
% http://www.latextemplates.com
%
% Original author:
% Ted Pavlic (http://www.tedpavlic.com)
%
% Note:
% The \lipsum[#] commands throughout this template generate dummy text
% to fill the template out. These commands should all be removed when
% writing assignment content.
%
%%%%%%%%%%%%%%%%%%%%%%%%%%%%%%%%%%%%%%%%%

%----------------------------------------------------------------------------------
%   PACKAGES AND OTHER DOCUMENT CONFIGURATIONS
%----------------------------------------------------------------------------------

\documentclass{article}

\usepackage[utf8]{inputenc}
\usepackage[spanish]{babel}
\usepackage{fancyhdr} % Required for custom headers
\usepackage{lastpage} % Required to determine the last page for the footer
\usepackage{graphicx} % Required to insert images
\usepackage{tikz}
\usepackage[export]{adjustbox}
\usepackage{enumitem}
\usepackage{environ}
\usepackage{multicol}
\usepackage{hyperref}
\usepackage[font=small]{caption}

\usepackage{listings}


%\usepackage{pifont}
\selectlanguage{spanish}
\addto\extrasspanish{%
    \def\figureautorefname{Figura}%
}
\newcommand{\myarrow}{\ding{223}}
\newcommand{\cmark}{\ding{51}}
\newcommand{\xmark}{\ding{55}}

% Margins
\topmargin=-0.45in
\evensidemargin=0in
\oddsidemargin=0in
\textwidth=6.5in
\textheight=9.0in
\headsep=0.25in

\linespread{1.1} % Line spacing

% Set up the header and footer
\pagestyle{fancy}
\lhead{\small \hmwkClass: \hmwkTitle} % Top left header
\chead{} % Top center header
\rhead{\small \hmwkAuthorName} % Top right header
\lfoot{} % Bottom left footer
\cfoot{} % Bottom center footer
\rfoot{Página\ \thepage\ de\ \pageref{LastPage}} % Bottom right footer
\renewcommand\headrulewidth{0.4pt} % Size of the header rule
\renewcommand\footrulewidth{0.4pt} % Size of the footer rule

\setlength\parindent{0pt} % Removes all indentation from paragraphs
\setlength{\multicolsep}{6.0pt plus 2.0pt minus 1.5pt} % 50% of original values

\newlength\widest
\makeatletter
\NewEnviron{ldescription}{%
    \vbox{%
        \global\setlength\widest{0pt}%
        \def\item[##1]{%
            \settowidth\@tempdima{\textbf{##1}}%
            \ifdim \@tempdima>\widest \global\setlength\widest{\@tempdima} \fi%
        }%
        \setbox0=\hbox{\BODY}%
    }
    \begin{description}[leftmargin=\dimexpr\widest+0.5em\relax,labelindent=0pt, labelwidth=\widest]
        \BODY
\end{description}%
}
\makeatother

%----------------------------------------------------------------------------------
%   NAME AND CLASS SECTION
%----------------------------------------------------------------------------------

\newcommand{\hmwkTitle}{Práctica\ 3} % Assignment title
\newcommand{\hmwkClass}{Sistemas Informáticos I} % Course/class
\newcommand{\hmwkClassTime}{10:30am} % Class/lecture time
\newcommand{\hmwkAuthorName}{\small Sergio Fuentes de Uña | Daniel Perdices Burrero} % Your name

%----------------------------------------------------------------------------------
%   TITLE PAGE
%----------------------------------------------------------------------------------

\title{
    \vspace{2in}
    \textmd{\textbf{\hmwkClass:\ \hmwkTitle}}\\
    \vspace{0.1in}
    \begin{center}
        \begin{tikzpicture}
            \begin{scope}
                \clip [rounded corners=12pt] (0,0) rectangle coordinate (centerpoint) (160pt,40pt);
                \node [inner sep=0pt] at (centerpoint) {\includegraphics[width=160pt]{img/betabet}};
            \end{scope}
        \end{tikzpicture}
    \end{center}
    \vspace{3in}
}

\author{\textbf{\hmwkAuthorName}}

%----------------------------------------------------------------------------------
\lstdefinestyle{sql}
{language=sql,
}
\begin{document}

\maketitle

%----------------------------------------------------------------------------------
%   TABLE OF CONTENTS
%----------------------------------------------------------------------------------

%\setcounter{tocdepth}{1} % Uncomment this line if you don't want subsections listed in the ToC

\newpage
\tableofcontents
\newpage
\section{Análisis y resultados de la base de datos proporcionada}

\subsection{Esquema de las tablas proporcionadas}
Leyenda: \textit{\underline{primary key}} \texttt{$\Uparrow$}:\textit{foreign key}

\begin{multicols}{2}
\begin{itemize}
    \item\texttt{customers}
        \begin{itemize}
            \item\texttt{\underline{customerid}} 
            \item\texttt{firstname}
            \item\texttt{lastname}
            \item\texttt{address1}
            \item\texttt{address2}
            \item\texttt{city}
            \item\texttt{state}
            \item\texttt{zip}
            \item\texttt{country}
            \item\texttt{region}
            \item\texttt{email}
            \item\texttt{phone}
            \item\texttt{creditcardtype}
            \item\texttt{creditcard}
            \item\texttt{creditcardexpiration}
            \item\texttt{username}
            \item\texttt{password}
            \item\texttt{age}
            \item\texttt{credit}
            \item\texttt{gender}
        \end{itemize}
    \item\texttt{clientorders}
        \begin{itemize}
            \item\texttt{customerid}
            \item\texttt{date}
            \item\texttt{orderid}
        \end{itemize}
    \columnbreak
    \item\texttt{clientbets}
        \begin{itemize}
            \item\texttt{customerid$\Uparrow$}
            \item\texttt{optionid}$\Uparrow$
            \item\texttt{betid}$\Uparrow$
            \item\texttt{orderid}
            \item\texttt{bet}
            \item\texttt{ratio}
            \item\texttt{outcome}
        \end{itemize}
    \item\texttt{bets}
        \begin{itemize}
            \item\texttt{\underline{betid}}
            \item\texttt{betcloses}
            \item\texttt{category}
            \item\texttt{betdesc}
            \item\texttt{winneropt}
        \end{itemize}
    \item\texttt{options}
        \begin{itemize}
            \item\texttt{\underline{optionid}}
            \item\texttt{optiondesc}
            \item\texttt{categoria}
            \item\texttt{cuota}
        \end{itemize}
    \item\texttt{optionbet}
        \begin{itemize}
            \item\texttt{optionid}$\Uparrow$
            \item\texttt{betid}$\Uparrow$
            \item\texttt{ratio}
            \item\texttt{optiondesc}
        \end{itemize}
\end{itemize}
\end{multicols}
\newpage
\subsection{Organización de las tablas}
La base de datos consta de las siguientes tablas:
\begin{ldescription}
    \item[$\bullet$ \texttt{customers}]
        Clientes de la página de apuestas.
    \item[$\bullet$ \texttt{clientorders}]
        Carritos de los clientes.
    \item[$\bullet$ \texttt{clientbets}]
        Apuestas realizadas por los clientes.
    \item[$\bullet$ \texttt{bets}]
        Apuestas disponibles.
    \item[$\bullet$ \texttt{options}]
        Entidades deportivas.
    \item[$\bullet$ \texttt{optionbet}]
        Entidades concretas que se involucran en cada apuesta
\end{ldescription}
Estas tablas se encuentran relacionadas a traves del siguiente diagrama Entidad-Relación
\smallbreak
\begin{minipage}{\linewidth}
    \centering
    \captionsetup{type=figure}
    \includegraphics[width=\linewidth]{img/db_old.png}
    \caption{Diagrama E-R (sin atributos) de la base de datos proporcionada}
    \label{fig:fig1}
\end{minipage}
\subsection{Diseño inicial de las tablas}
Con lo proporcionado, se pueden ver varias claras decisiones de diseño tomadas a la hora de elaborar la base de datos que se ha proporcionado:
\begin{enumerate}
    \item El atributo \texttt{customerid} de la tabla \texttt{clientbets} es redundante con el atributo de igual nombre de la tabla \texttt{clientorders}.
    \item El atributo \texttt{category} de la tabla \texttt{bets} y el atributo \texttt{categoria} de la tabla \texttt{options} son redundantes de igual manera, además, en la tabla las categorías se suelen repetir, por tanto se gasta mucho espacio de almacenamiento en guardar la misma cadena varias ocasiones.
    \item El atributo \texttt{optiondesc} del mismo modo aparece en dos tablas, \texttt{options} y \texttt{optionbet}, aunque se podría ver como el mismo.
\end{enumerate}
Muchas de estas repeticiones, especialmente la del \texttt{customerid} son bastante discutibles pues la repetición del dato mejora el rendimiento de la base de datos.
\subsection{Propuesta de diseño de las tablas}
Se han realizado los siguientes cambios en las tablas para normalizar la base de datos evitando futuras incosistencias. Se han añadido restricciones para evitar registros erroneos.
\begin{itemize}
    \item\texttt{customers:}
        \begin{itemize}
            \item Se agrega la restricción de email válido.
            \item Se agrega la restricción de edad válida.
            \item Se agrega la restricción de código postal válida.
            \item Se agrega la restricción de tarjeta de crédito válida.
            \item Se agrega la restricción de teléfono válido.
        \end{itemize}
    \item\texttt{clientorders:}
        \begin{itemize}
            \item Se agrega la restricción de orderid como primary key.
            \item Se agrega la restricción de customerid como foreign key.
            \item Se activa el borrado en cascada si se elimina el cliente (customerid).
        \end{itemize}
    \item\texttt{clientbets:}
        \begin{itemize}
            \item Se agrega la restricción de orderid como foreign key.
            \item Se elimina la columna customerid.
            \item Se activa el borrado en cascada si se elimina el carrito (orderid).
        \end{itemize}
    \item\texttt{bets}:
        \begin{itemize}
            \item Se elimina el campo category.
            \item Se agrega el campo categoryid como foreign key.
            \item Se agrega la restricción de winneropt como foreign key.
        \end{itemize}
    \item\texttt{options}
        \begin{itemize}
            \item Se elimina el campo categoria.
            \item Se agrega el campo categoryid como foreign key.
        \end{itemize}
    \item\texttt{optionbet}
        \begin{itemize}
            \item Se elimina la columna optiondesc.
        \end{itemize}
    \item\texttt{categories}
        \begin{itemize}
            \item Se agrega la tabla.
            \item Se agrega el campo categoryid como primary key.
            \item Se agrega el campo categorystring.
        \end{itemize}
\end{itemize}
\newpage
A continuación se muestra el diagrama E-R (sin atributos por legibilidad) del estado final.
\smallbreak
\begin{minipage}{\linewidth}
    \centering
    \captionsetup{type=figure}
    \includegraphics[width=\linewidth]{img/db_new_2}
    \caption{Diagrama E-R de la base de datos final}
    \label{fig:fig2}
\end{minipage}
Se muestra la nueva tabla \texttt{categories} así como algún detalle que se ha puntualizado. Un ejemplo claro es la entidad débil (\textit{weak entity}) de \texttt{clientbets}, las cuales solo pueden existir dentro de un \texttt{order}. Se podría decir de la misma manera que \texttt{order} es una entidad débil dependiente del \texttt{customer}, sin embargo, para aumentar la libertad de diseño del sistema web a la hora de gestionar el order, esta restricción no se impone.
Todos estos cambios los realiza el fichero \texttt{actualiza.sql}

\section{Consultas SQL}
El objetivo de esta práctica ha sido implementar un conjunto de funciones y \textit{triggers} que faciliten el mantenimiento y manejo de la base de datos y de la aplicación web.
\subsection{Análisis de las consultas realizadas}
A continuación se relatan los resultados obtenidos así como un resumen del rendimiento de las consultas.
\subsubsection{setTotalAmount}

\textbf{Comprobación de los resultados}

Se muestra el primer pedido previa la ejecución del ejercicio
\lstset{basicstyle=\small,style=sql}
\begin{lstlisting}[style=sql]
 customerid |          date          | orderid | totalamount
------------+------------------------+---------+-------------
        693 | 2012-12-02 17:23:39+01 |       1 |
\end{lstlisting}
\begin{lstlisting}[style=sql]
 optionid |  bet  | ratio | outcome  | betid | orderid
----------+-------+-------+----------+-------+---------
      117 | 79.71 |  9.01 |          |   164 |       1
       47 | 23.30 |  8.42 |          |  2372 |       1
       51 | 74.58 |  5.92 |          |  2418 |       1
       86 | 94.57 |  9.69 |          |  2438 |       1
\end{lstlisting} 
Tras ejecutar el ejercicio resulta:
\begin{lstlisting}[style=sql]

 customerid |          date          | orderid | totalamount
------------+------------------------+---------+-------------
        693 | 2012-12-02 17:23:39+01 |       1 |      272.16
\end{lstlisting}
Por lo que se observa que la consulta se realiza correctamente
$$\;$$

\textbf{Rendimiento y optimización}

El rendimiento original de la consulta
\lstset{basicstyle=\small,style=sql}
\begin{lstlisting}[style=sql]
 Update on clientorders  (cost=97847.15..112110.70 rows=118766 width=146)
   ->  Hash Join  (cost=97847.15..112110.70 rows=118766 width=146)
         Hash Cond: (aux.id = clientorders.orderid)
         ->  Subquery Scan on aux  (cost=90991.97..98686.49 rows=118766 
         width=96)
               ->  GroupAggregate  (cost=90991.97..97498.83 rows=118766 
               width=10)
                     Group Key: clientbets.orderid
                     ->  Sort  (cost=90991.97..92666.07 rows=669638 width=10)
                           Sort Key: clientbets.orderid
                           ->  Seq Scan on clientbets  (cost=0.00..14749.38 
                           rows=669638 width=10)
         ->  Hash  (cost=3536.30..3536.30 rows=149030 width=54)
               ->  Seq Scan on clientorders  (cost=0.00..3536.30 rows=149030 
               width=54)

\end{lstlisting}
Para optimizar esta consulta, que pese a no tardar un tiempo excesivo, se detectan lecturas secuenciales, se crean índices.

Rendimiento tras la creación del índice sobre \texttt{clientbets(orderid)}
\begin{lstlisting}[style=sql]
 Update on clientorders  (cost=6855.60..69056.69 rows=118766 width=146)
   ->  Hash Join  (cost=6855.60..69056.69 rows=118766 width=146)
         Hash Cond: (aux.id = clientorders.orderid)
         ->  Subquery Scan on aux  (cost=0.42..55632.49 rows=118766 width=96)
               ->  GroupAggregate  (cost=0.42..54444.83 rows=118766 width=10)
                     Group Key: clientbets.orderid
                     ->  Index Scan using clbets_oid_idx on clientbets  (
                     cost=0.42..49612.05 rows=669640 width=10)
         ->  Hash  (cost=3536.30..3536.30 rows=149030 width=54)
               ->  Seq Scan on clientorders  (cost=0.00..3536.30 rows=149030
                width=54)
\end{lstlisting}

Rendimiento tras la creación del índice sobre \texttt{clientorders(orderid)}
\begin{lstlisting}[style=sql]
 Update on clientorders  (cost=0.84..69557.52 rows=118766 width=146)
   ->  Merge Join  (cost=0.84..69557.52 rows=118766 width=146)
         Merge Cond: (aux.id = clientorders.orderid)
         ->  Subquery Scan on aux  (cost=0.42..55632.49 rows=118766 width=96)
               ->  GroupAggregate  (cost=0.42..54444.83 rows=118766 width=10)
                     Group Key: clientbets.orderid
                     ->  Index Scan using clbets_oid_idx on clientbets  (
                     cost=0.42..49612.05 rows=669640 width=10)
         ->  Index Scan using clo_oid_idx on clientorders  (cost=0.42..12067.88 
         rows=149031 width=54)

\end{lstlisting}

Se observa una mejoría en el coste respecto al valor original sin índices. Hace falta recordar que los índices pueden ralentizar las inserciones o actualizaciones por ser necesario actualizar los índices en estos casos.

\subsubsection{setOutcomeBets}

\textbf{Comprobación de los resultados}

Miramos de nuevo el mismo pedido y los winneropt
\begin{lstlisting}[style=sql]
  bet  | outcome | winneropt | optionid
-------+---------+-----------+----------
 79.71 |         |       120 |      117
 23.30 |         |        47 |       47
 74.58 |         |        51 |       51
 94.57 |         |        86 |       86
 \end{lstlisting}
Tras ejecutar el ejercicio debería resultar que si winneropt=optionid, entonces el outcome debería ser ratio·bet, en caso de que no sea null pero sean distintos los valores, el outcome debería ser 0
\begin{lstlisting}[style=sql]
  bet  | ratio | outcome  | winneropt | optionid
-------+-------+----------+-----------+----------
 79.71 |  9.01 |        0 |       120 |      117
 23.30 |  8.42 | 196.1860 |        47 |       47
 74.58 |  5.92 | 441.5136 |        51 |       51
 94.57 |  9.69 | 916.3833 |        86 |       86
 \end{lstlisting}

$$\;$$

\textbf{Rendimiento y optimización}

En este caso la única lectura secuencial de las tablas es necesaria para la actualización de todos los valores. Se observa otra lectura secuencial pero de coste mínimo.

\begin{lstlisting}[style=sql]
 Update on clientbets  (cost=333.20..20155.54 rows=4051 width=36)
   ->  Hash Join  (cost=333.20..20155.54 rows=4051 width=36)
         Hash Cond: ((clientbets.optionid = bets.winneropt) AND (
         clientbets.betid = bets.betid))
         ->  Seq Scan on clientbets  (cost=0.00..14749.40 rows=669640 width=30)
         ->  Hash  (cost=220.88..220.88 rows=7488 width=14)
               ->  Seq Scan on bets  (cost=0.00..220.88 rows=7488 width=14)
\end{lstlisting}

\subsubsection{setOrderTotalOutcome}
Como no se sabía si se quería un procedimiento almacenado que dado un orderid, calcule el outcome de ese carrito o se quería un procedimiento que lo calculase sobre todos los carritos, se dan ambas soluciones:
\begin{enumerate}
\item setOrderTotalOutcome(orderid\_arg integer)
\item setTotalOutcome()
\end{enumerate}
\textbf{Comprobación de los resultados}
De nuevo se observa el primer pedido sin outcome y simplemente hace falta fijarse en los outcome del ejercicio anterior y sumarlos para saber el resultado.
\begin{lstlisting}[style=sql]

 customerid |          date          | orderid | totalamount | totaloutcome
------------+------------------------+---------+-------------+--------------
        693 | 2012-12-02 17:23:39+01 |       1 |      272.16 |
\end{lstlisting}
Se comprueba que efectivamente el resultado coincide.
\begin{lstlisting}[style=sql]

 customerid |          date          | orderid | totalamount | totaloutcome
------------+------------------------+---------+-------------+--------------
        693 | 2012-12-02 17:23:39+01 |       1 |      272.16 |    1554.0829
\end{lstlisting}

$$\;$$

\textbf{Rendimiento y optimización}
Se muestra ya el coste tras construir los índices oportunos.
\begin{lstlisting}[style=sql]
 Update on clientbets  (cost=333.20..20155.54 rows=4051 width=36)
   ->  Hash Join  (cost=333.20..20155.54 rows=4051 width=36)
         Hash Cond: ((clientbets.optionid = bets.winneropt) AND (
         clientbets.betid = bets.betid))
         ->  Seq Scan on clientbets  (cost=0.00..14749.40 rows=669640 width=30)
         ->  Hash  (cost=220.88..220.88 rows=7488 width=14)
               ->  Seq Scan on bets  (cost=0.00..220.88 rows=7488 width=14)
\end{lstlisting}

\subsection{\textit{Triggers}}
Se incluyen los tres triggers solicitados:
\begin{enumerate}
    \item updBets: actualiza el campo outcome de las apuestas realizadas por un cliente una vez se ha designado un ganador.
    \item updOrders: actualiza el campo totalamount y totaloutcome del carrito cuando se modifica por el \textit{trigger} anterior.
    \item updCredit: actualiza el campo credit del cliente cuando un carrito se cierra o se modifica uno cerrado (asignación de ganadores).
\end{enumerate}

El correcto funcionamiento de estos \textit{triggers} se puede observar mediante los ficheros \texttt{tests.sql} y \texttt{tests2   .sql}. Este fichero no incluye el borrado del carrito de ya existir luego se recomienda encarecidamente borrar el carrito que se crea durante el test en caso de querer ejecutarlo varias veces. El test muestra varias inserciones o borrados de un carrito y como el credit se va actualizando según cambia el estado del carrito.

Parte del código se ha reutilizado del apartado anterior.

\textbf{Nota:} Debido a que parte de la funcionalidad de los apartados anteriores es realizada por los \textit{triggers}, no es nada recomendable ejecutar de nuevo las actualizaciones, pues de hacerlo se actualizarán todas las filas de las tablas generando tantas ejecuciones de los \textit{triggers} como filas tienen las tablas.

\section{Adaptación a la aplicación Web}
Para esta tarea se ha realizado un fichero SQL llamado \texttt{adapta.sql} que realiza ciertas adaptaciones sobre la base de datos proporcionada para dar menos relevancia a los atributos de la base de datos que no tienen relación con la aplicación o una función dentro de la misma.

\subsection{Cambios realizados}
Los cambios realizados son:

\begin{itemize}
\item Quitar restricción \texttt{not null} en atributo \texttt{firstname} de la tabla \texttt{customers} 
\item Quitar restricción \texttt{not null} en atributo \texttt{lastname} de la tabla \texttt{customers} 
\item Quitar restricción \texttt{not null} en atributo \texttt{address1} de la tabla \texttt{customers} 
\item Quitar restricción \texttt{not null} en atributo \texttt{city} de la tabla \texttt{customers} 
\item Quitar restricción \texttt{not null} en atributo \texttt{country} de la tabla \texttt{customers} 
\item Quitar restricción \texttt{not null} en atributo \texttt{region} de la tabla \texttt{customers} 
\item Quitar restricción \texttt{not null} en atributo \texttt{creditcardtype} de la tabla \texttt{customers} 
\item Cambiar los atributos de actualización secuencial de las ids de todas las tablas, de manera que al insertan se inserten automáticamente con id = max (ids)
\end{itemize}

\subsection{Funcionalidad de la página web implementada}

Se ha implementado una interacción de la web con la base de datos casi completa:
\begin{itemize}
\item Se ha integrado con la parte de gestión y creación de usuarios
\item Se ha integrado con la parte de visualización de las apuestas y el sistema creación y apuestas.
\item Se ha integrado la gestión del carrito con la base de datos, de manera que el carrito se guarda en la base de datos.
\item Se ha implementado un sistema de carga incremental mediante AJAX para evitar el envío de un gran volumen de datos en la carga de la página principal.
\end{itemize}

El objetiva final logrado ha sido integrar totalmente la página desarrollada con la base de datos, de tal manera que la página es totalmente funcional y todo el manejo se realiza usando la base de datos.

\end{document}
